\chapter*{Conclusion and future outlook}

\label{sec:conclusion}

\clearpage

\section*{Conclusion}
In this thesis, we have developed and evaluated a robust end-to-end pipeline for automated skin lesion classification, based on a Mixture-of-Experts (MoE) architecture with Transformer-based feature extractors. Leveraging a balanced, augmented HAM10000 dataset and mixed-precision training on consumer-grade hardware, our model achieved 93\% overall accuracy and 84\% balanced accuracy on a held-out test set. Key innovations include dynamic expert routing, a load-balancing regularizer to ensure equitable expert utilization, and deployment-ready export to TorchScript/ONNX formats. These results demonstrate the viability of the MoE approach for dermatological image analysis, outperforming or matching state-of-the-art benchmarks while maintaining deployment flexibility.

\section*{Future outlook}
Building on these findings, future work will focus on transfer to clinical settings and advanced edge deployments. We plan to:
\begin{itemize}
  \item Integrate additional dermoscopic and non-dermoscopic datasets to improve generalization across imaging devices and populations.
  \item Incorporate patient metadata (age, lesion location, history) into the gating network to enhance diagnostic context.
  \item Evaluate post-training quantization and structured pruning on Coral Dev Boards with Edge TPUs for real-time, low-power inference.
  \item Explore semi-supervised and self-supervised techniques to leverage unlabeled clinical images and reduce annotation costs.
  \item Conduct prospective clinical validation studies to assess model impact on diagnostic workflow and patient outcomes.
\end{itemize}

%%% Local Variables: 
%%% mode: latex
%%% TeX-master: "isae-report-template"
%%% End:

